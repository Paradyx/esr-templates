%%%
% Silbentrennung
\usepackage[ngerman]{babel}

%%%
% Fonts and typesetting
\usepackage{mathspec} %includes fontspec
%\setmainfont[BoldFont={Source S Pro Semibold}, ItalicFont={Source Sans Pro Italic}, BoldItalicFont={Source Sans Pro Semibold Italic}]{Source Sans Pro}
\usepackage{bm}
\usepackage{fontspec}
\setmainfont[
UprightFont={* Light},
BoldFont={* Semibold},
ItalicFont={* Light Italic},
BoldItalicFont={* Semibold Italic}
]
{Source Serif Pro}

\setsansfont[
UprightFont={* Light},
BoldFont={* Semibold},
ItalicFont={* Light Italic},
BoldItalicFont={* Semibold Italic}]
{Source Sans Pro}%
[Scale=MatchLowercase]

\setmonofont[
BoldFont={Source Code Pro Semibold}]
{Source Code Pro Light}%
[Scale=MatchLowercase]

\setmathfont(Digits,Latin)[
UprightFont={* Light},
BoldFont={* Semibold},
ItalicFont={* Light Italic},
BoldItalicFont={* Semibold Italic}
]
{Source Serif Pro}

\setmathfont(Greek)[
UprightFont={* Light},
BoldFont={* Semibold},
AutoFakeSlant=0.15
]
{Source Serif Pro}

% Fix mathspec before amsmath bug
% https://tex.stackexchange.com/questions/85696/what-causes-this-strange-interaction-between-glossaries-and-amsmath
\makeatletter % undo the wrong changes made by mathspec
\let\RequirePackage\original@RequirePackage
\let\usepackage\RequirePackage
\makeatother
%
%%%

%%%
% Memoir Class Page formatting
%
% Change line witdh to new font
\setlxvchars \setxlvchars
% 5mm binding tollerance
%\setbinding{5mm}
\settypeblocksize{600pt}{1.2\lxvchars}{*}
\setlrmargins{*}{*}{1}
\setulmargins{4cm}{*}{*}
\setheadfoot{2\onelineskip}{2\onelineskip}
\setmarginnotes{17pt}{3.5cm}{\onelineskip}
\checkandfixthelayout
%%%
% Superscripts for 1st, 2nd, 3rd, 4th
% \nth{3}
\usepackage[super]{nth}
%
%%%

%%%
% Colors
\usepackage{xcolor}
\definecolor{librarium-darkgreen}{RGB}{115, 173, 39}
\definecolor{librarium-lightgreen}{RGB}{147, 197, 75}
\definecolor{librarium-darkbrown}{RGB}{163, 141, 91}
\definecolor{librarium-lightbrown}{RGB}{231, 230, 226}

%
%%%

%%%
% Division Styles
% Chapter style definitions
\makechapterstyle{karticle} {
	\chapterstyle{article}
	\renewcommand\chaptitlefont{%
		\normalfont%
		\huge%
		\bfseries%
		\color{rwth-75}%
		\raggedright
	}%
}
% Chapter Style
%\chapterstyle{hansen} % use for nice chapter
%\chapterstyle{karticle} % use for articles

\makechapterstyle{librariumordnung}{}
\chapterstyle{librariumordnung}

% Section Style
\setsecheadstyle{\Large\bfseries\raggedright}
%
%%%

%%%
% Acronyms
\usepackage[acronym,nomain,nonumberlist,nopostdot,numberedsection]{glossaries}
\usepackage{glossary-longragged}
\input{acronyms}
\makeglossaries
%
%%%

%%%
% Grafikeinbindungen
\usepackage{graphicx}
% Relocate default path
\graphicspath{{\getgraphicspath}}
\newcommand{\getgraphicspath}{../figures/}
%
%%%

%%%
% Tikz
%\usepackage{tikz}
%\input{../tikz/tikz-styles}
%
%%%

%%%
% Listings
\usepackage{listings}
%
%%%

%%%%
% Tables
\DisemulatePackage{tabularx}
\usepackage{multirow,tabu, makecell}
%
%%%

%%%
% Quotes
% Zitate in \enquote{...} setzen, dann werden automatisch die richtigen Anführungszeichen verwendet.
\usepackage{csquotes}
%
%%%

%%%
%schoene TODOs
\usepackage{todonotes}
\newcommand{\xtodo}[1]{\todo[color=black!7]{#1}\xspace}
\newcommand{\itodo}[2][]{\todo[inline,color=green!5,#1]{#2}}
%
%%%

%%%
% Abbreviations
\usepackage{xspace}
\newcommand*{\eg}[0]{e.g.\@\xspace}
\newcommand*{\ie}[0]{i.e.\@\xspace}
\newcommand*{\wrt}[0]{w.r.t.\@\xspace}
\newcommand*{\cf}[0]{c.f.\@\xspace}
\newcommand*{\zb}[0]{z.B.\@\xspace}

\makeatletter
\newcommand*{\etc}{%
  \@ifnextchar{.}%
  {etc}%
  {etc.\@\xspace}%
}
\makeatother
%%%

%%%
% SI Einheiten
%\usepackage[binary-units]{siunitx}
%\sisetup{
%  detect-all, % select font from suroundings
%  range-phrase = --,
%  list-final-separator = {, },
%  group-digits = integer,
%  group-minimum-digits = 4
%}
%
%%%
